%!TeX root=../signtop.tex
\chapter{The Great Agra Treasure}
\lettrine[lines=4]{O}{ur} captive sat in the cabin opposite to the iron box which he had done so much and waited so long to gain. He was a sunburned, reckless-eyed fellow, with a network of lines and wrinkles all over his mahogany features, which told of a hard, open-air life. There was a singular prominence about his bearded chin which marked a man who was not to be easily turned from his purpose. His age may have been fifty or thereabouts, for his black, curly hair was thickly shot with grey. His face in repose was not an unpleasing one, though his heavy brows and aggressive chin gave him, as I had lately seen, a terrible expression when moved to anger. He sat now with his handcuffed hands upon his lap, and his head sunk upon his breast, while he looked with his keen, twinkling eyes at the box which had been the cause of his ill-doings. It seemed to me that there was more sorrow than anger in his rigid and contained countenance. Once he looked up at me with a gleam of something like humour in his eyes.

»Well, Jonathan Small,« said Holmes, lighting a cigar, »I am sorry that it has come to this.«

»And so am I, sir,« he answered, frankly. »I don't believe that I can swing over the job. I give you my word on the book that I never raised hand against Mr Sholto. It was that little hell-hound Tonga who shot one of his cursed darts into him. I had no part in it, sir. I was as grieved as if it had been my blood-relation. I welted the little devil with the slack end of the rope for it, but it was done, and I could not undo it again.«

»Have a cigar,« said Holmes; »and you had best take a pull out of my flask, for you are very wet. How could you expect so small and weak a man as this black fellow to overpower Mr Sholto and hold him while you were climbing the rope?«

»You seem to know as much about it as if you were there, sir. The truth is that I hoped to find the room clear. I knew the habits of the house pretty well, and it was the time when Mr Sholto usually went down to his supper. I shall make no secret of the business. The best defence that I can make is just the simple truth. Now, if it had been the old major I would have swung for him with a light heart. I would have thought no more of knifing him than of smoking this cigar. But it's cursed hard that I should be lagged over this young Sholto, with whom I had no quarrel whatever.«

»You are under the charge of Mr Athelney Jones, of Scotland Yard. He is going to bring you up to my rooms, and I shall ask you for a true account of the matter. You must make a clean breast of it, for if you do I hope that I may be of use to you. I think I can prove that the poison acts so quickly that the man was dead before ever you reached the room.«

»That he was, sir. I never got such a turn in my life as when I saw him grinning at me with his head on his shoulder as I climbed through the window. It fairly shook me, sir. I'd have half killed Tonga for it if he had not scrambled off. That was how he came to leave his club, and some of his darts too, as he tells me, which I dare say helped to put you on our track; though how you kept on it is more than I can tell. I don't feel no malice against you for it. But it does seem a queer thing,« he added, with a bitter smile, »that I who have a fair claim to nigh upon half a million of money should spend the first half of my life building a breakwater in the Andamans, and am like to spend the other half digging drains at Dartmoor. It was an evil day for me when first I clapped eyes upon the merchant Achmet and had to do with the Agra treasure, which never brought anything but a curse yet upon the man who owned it. To him it brought murder, to Major Sholto it brought fear and guilt, to me it has meant slavery for life.«

At this moment Athelney Jones thrust his broad face and heavy shoulders into the tiny cabin. »Quite a family party,« he remarked. »I think I shall have a pull at that flask, Holmes. Well, I think we may all congratulate each other. Pity we didn't take the other alive; but there was no choice. I say, Holmes, you must confess that you cut it rather fine. It was all we could do to overhaul her.«

»All is well that ends well,« said Holmes. »But I certainly did not know that the \textit{Aurora} was such a clipper.«

»Smith says she is one of the fastest launches on the river, and that if he had had another man to help him with the engines we should never have caught her. He swears he knew nothing of this Norwood business.«

»Neither he did,« cried our prisoner,—»not a word. I chose his launch because I heard that she was a flier. We told him nothing, but we paid him well, and he was to get something handsome if we reached our vessel, the \textit{Esmeralda}, at Gravesend, outward bound for the Brazils.«

»Well, if he has done no wrong we shall see that no wrong comes to him. If we are pretty quick in catching our men, we are not so quick in condemning them.« It was amusing to notice how the consequential Jones was already beginning to give himself airs on the strength of the capture. From the slight smile which played over Sherlock Holmes's face, I could see that the speech had not been lost upon him.

»We will be at Vauxhall Bridge presently,« said Jones, »and shall land you, Dr Watson, with the treasure-box. I need hardly tell you that I am taking a very grave responsibility upon myself in doing this. It is most irregular; but of course an agreement is an agreement. I must, however, as a matter of duty, send an inspector with you, since you have so valuable a charge. You will drive, no doubt?«

»Yes, I shall drive.«

»It is a pity there is no key, that we may make an inventory first. You will have to break it open. Where is the key, my man?«

»At the bottom of the river,« said Small, shortly.

»Hum! There was no use your giving this unnecessary trouble. We have had work enough already through you. However, doctor, I need not warn you to be careful. Bring the box back with you to the Baker Street rooms. You will find us there, on our way to the station.«

They landed me at Vauxhall, with my heavy iron box, and with a bluff, genial inspector as my companion. A quarter of an hour's drive brought us to Mrs Cecil Forrester's. The servant seemed surprised at so late a visitor. Mrs Cecil Forrester was out for the evening, she explained, and likely to be very late. Miss Morstan, however, was in the drawing-room: so to the drawing-room I went, box in hand, leaving the obliging inspector in the cab.

She was seated by the open window, dressed in some sort of white diaphanous material, with a little touch of scarlet at the neck and waist. The soft light of a shaded lamp fell upon her as she leaned back in the basket chair, playing over her sweet, grave face, and tinting with a dull, metallic sparkle the rich coils of her luxuriant hair. One white arm and hand drooped over the side of the chair, and her whole pose and figure spoke of an absorbing melancholy. At the sound of my foot-fall she sprang to her feet, however, and a bright flush of surprise and of pleasure coloured her pale cheeks.

»I heard a cab drive up,« she said. »I thought that Mrs Forrester had come back very early, but I never dreamed that it might be you. What news have you brought me?«

»I have brought something better than news,« said I, putting down the box upon the table and speaking jovially and boisterously, though my heart was heavy within me. »I have brought you something which is worth all the news in the world. I have brought you a fortune.«

She glanced at the iron box. »Is that the treasure, then?« she asked, coolly enough.

»Yes, this is the great Agra treasure. Half of it is yours and half is Thaddeus Sholto's. You will have a couple of hundred thousand each. Think of that! An annuity of ten thousand pounds. There will be few richer young ladies in England. Is it not glorious?«

I think that I must have been rather overacting my delight, and that she detected a hollow ring in my congratulations, for I saw her eyebrows rise a little, and she glanced at me curiously.

»If I have it,« said she, »I owe it to you.«

»No, no,« I answered, »not to me, but to my friend Sherlock Holmes. With all the will in the world, I could never have followed up a clue which has taxed even his analytical genius. As it was, we very nearly lost it at the last moment.«

»Pray sit down and tell me all about it, Dr Watson,« said she.

I narrated briefly what had occurred since I had seen her last,—Holmes's new method of search, the discovery of the \textit{Aurora}, the appearance of Athelney Jones, our expedition in the evening, and the wild chase down the Thames. She listened with parted lips and shining eyes to my recital of our adventures. When I spoke of the dart which had so narrowly missed us, she turned so white that I feared that she was about to faint.

»It is nothing,« she said, as I hastened to pour her out some water. »I am all right again. It was a shock to me to hear that I had placed my friends in such horrible peril.«

»That is all over,« I answered. »It was nothing. I will tell you no more gloomy details. Let us turn to something brighter. There is the treasure. What could be brighter than that? I got leave to bring it with me, thinking that it would interest you to be the first to see it.«

»It would be of the greatest interest to me,« she said. There was no eagerness in her voice, however. It had struck her, doubtless, that it might seem ungracious upon her part to be indifferent to a prize which had cost so much to win.

»What a pretty box!« she said, stooping over it. »This is Indian work, I suppose?«

»Yes; it is Benares metal-work.«

»And so heavy!« she exclaimed, trying to raise it. »The box alone must be of some value. Where is the key?«

»Small threw it into the Thames,« I answered. »I must borrow Mrs Forrester's poker.« There was in the front a thick and broad hasp, wrought in the image of a sitting Buddha. Under this I thrust the end of the poker and twisted it outward as a lever. The hasp sprang open with a loud snap. With trembling fingers I flung back the lid. We both stood gazing in astonishment. The box was empty!

No wonder that it was heavy. The iron-work was two-thirds of an inch thick all round. It was massive, well made, and solid, like a chest constructed to carry things of great price, but not one shred or crumb of metal or jewellery lay within it. It was absolutely and completely empty.

»The treasure is lost,« said Miss Morstan, calmly.

As I listened to the words and realised what they meant, a great shadow seemed to pass from my soul. I did not know how this Agra treasure had weighed me down, until now that it was finally removed. It was selfish, no doubt, disloyal, wrong, but I could realise nothing save that the golden barrier was gone from between us. »Thank God!« I ejaculated from my very heart.

She looked at me with a quick, questioning smile. »Why do you say that?« she asked.

»Because you are within my reach again,« I said, taking her hand. She did not withdraw it. »Because I love you, Mary, as truly as ever a man loved a woman. Because this treasure, these riches, sealed my lips. Now that they are gone I can tell you how I love you. That is why I said, »Thank God.««

»Then I say, »Thank God,« too,« she whispered, as I drew her to my side. Whoever had lost a treasure, I knew that night that I had gained one.