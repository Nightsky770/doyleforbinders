%!TeX root=../signtop.tex
\chapter{The Strange Story of Jonathan Small}
\lettrine[lines=4]{A}{} very patient man was that inspector in the cab, for it was a weary time before I rejoined him. His face clouded over when I showed him the empty box.

\zz
»There goes the reward!« said he, gloomily. »Where there is no money there is no pay. This night's work would have been worth a tenner each to Sam Brown and me if the treasure had been there.«

»Mr Thaddeus Sholto is a rich man,« I said. »He will see that you are rewarded, treasure or no.«

The inspector shook his head despondently, however. »It's a bad job,« he repeated; »and so Mr Athelney Jones will think.«

His forecast proved to be correct, for the detective looked blank enough when I got to Baker Street and showed him the empty box. They had only just arrived, Holmes, the prisoner, and he, for they had changed their plans so far as to report themselves at a station upon the way. My companion lounged in his arm-chair with his usual listless expression, while Small sat stolidly opposite to him with his wooden leg cocked over his sound one. As I exhibited the empty box he leaned back in his chair and laughed aloud.

»This is your doing, Small,« said Athelney Jones, angrily.

»Yes, I have put it away where you shall never lay hand upon it,« he cried, exultantly. »It is my treasure; and if I can't have the loot I'll take darned good care that no one else does. I tell you that no living man has any right to it, unless it is three men who are in the Andaman convict-barracks and myself. I know now that I cannot have the use of it, and I know that they cannot. I have acted all through for them as much as for myself. It's been the sign of four with us always. Well I know that they would have had me do just what I have done, and throw the treasure into the Thames rather than let it go to kith or kin of Sholto or of Morstan. It was not to make them rich that we did for Achmet. You'll find the treasure where the key is, and where little Tonga is. When I saw that your launch must catch us, I put the loot away in a safe place. There are no rupees for you this journey.«

»You are deceiving us, Small,« said Athelney Jones, sternly. »If you had wished to throw the treasure into the Thames it would have been easier for you to have thrown box and all.«

»Easier for me to throw, and easier for you to recover,« he answered, with a shrewd, sidelong look. »The man that was clever enough to hunt me down is clever enough to pick an iron box from the bottom of a river. Now that they are scattered over five miles or so, it may be a harder job. It went to my heart to do it, though. I was half mad when you came up with us. However, there's no good grieving over it. I've had ups in my life, and I've had downs, but I've learned not to cry over spilled milk.«

»This is a very serious matter, Small,« said the detective. »If you had helped justice, instead of thwarting it in this way, you would have had a better chance at your trial.«

»Justice!« snarled the ex-convict. »A pretty justice! Whose loot is this, if it is not ours? Where is the justice that I should give it up to those who have never earned it? Look how I have earned it! Twenty long years in that fever-ridden swamp, all day at work under the mangrove-tree, all night chained up in the filthy convict-huts, bitten by mosquitoes, racked with ague, bullied by every cursed black-faced policeman who loved to take it out of a white man. That was how I earned the Agra treasure; and you talk to me of justice because I cannot bear to feel that I have paid this price only that another may enjoy it! I would rather swing a score of times, or have one of Tonga's darts in my hide, than live in a convict's cell and feel that another man is at his ease in a palace with the money that should be mine.« Small had dropped his mask of stoicism, and all this came out in a wild whirl of words, while his eyes blazed, and the handcuffs clanked together with the impassioned movement of his hands. I could understand, as I saw the fury and the passion of the man, that it was no groundless or unnatural terror which had possessed Major Sholto when he first learned that the injured convict was upon his track.

»You forget that we know nothing of all this,« said Holmes quietly. »We have not heard your story, and we cannot tell how far justice may originally have been on your side.«

»Well, sir, you have been very fair-spoken to me, though I can see that I have you to thank that I have these bracelets upon my wrists. Still, I bear no grudge for that. It is all fair and above-board. If you want to hear my story I have no wish to hold it back. What I say to you is God's truth, every word of it. Thank you; you can put the glass beside me here, and I'll put my lips to it if I am dry.

I am a Worcestershire man myself,—born near Pershore. I dare say you would find a heap of Smalls living there now if you were to look. I have often thought of taking a look round there, but the truth is that I was never much of a credit to the family, and I doubt if they would be so very glad to see me. They were all steady, chapel-going folk, small farmers, well-known and respected over the country-side, while I was always a bit of a rover. At last, however, when I was about eighteen, I gave them no more trouble, for I got into a mess over a girl, and could only get out of it again by taking the Queen's shilling and joining the 3\textsuperscript{rd} Buffs, which was just starting for India.

I wasn't destined to do much soldiering, however. I had just got past the goose-step, and learned to handle my musket, when I was fool enough to go swimming in the Ganges. Luckily for me, my company sergeant, John Holder, was in the water at the same time, and he was one of the finest swimmers in the service. A crocodile took me, just as I was half-way across, and nipped off my right leg as clean as a surgeon could have done it, just above the knee. What with the shock and the loss of blood, I fainted, and should have drowned if Holder had not caught hold of me and paddled for the bank. I was five months in hospital over it, and when at last I was able to limp out of it with this timber toe strapped to my stump I found myself invalided out of the army and unfitted for any active occupation.

I was, as you can imagine, pretty down on my luck at this time, for I was a useless cripple though not yet in my twentieth year. However, my misfortune soon proved to be a blessing in disguise. A man named Abel White, who had come out there as an indigo-planter, wanted an overseer to look after his coolies and keep them up to their work. He happened to be a friend of our colonel's, who had taken an interest in me since the accident. To make a long story short, the colonel recommended me strongly for the post and, as the work was mostly to be done on horseback, my leg was no great obstacle, for I had enough knee left to keep good grip on the saddle. What I had to do was to ride over the plantation, to keep an eye on the men as they worked, and to report the idlers. The pay was fair, I had comfortable quarters, and altogether I was content to spend the remainder of my life in indigo-planting. Mr Abel White was a kind man, and he would often drop into my little shanty and smoke a pipe with me, for white folk out there feel their hearts warm to each other as they never do here at home.

Well, I was never in luck's way long. Suddenly, without a note of warning, the great mutiny broke upon us. One month India lay as still and peaceful, to all appearance, as Surrey or Kent; the next there were two hundred thousand black devils let loose, and the country was a perfect hell. Of course you know all about it, gentlemen,—a deal more than I do, very like, since reading is not in my line. I only know what I saw with my own eyes. Our plantation was at a place called Muttra, near the border of the Northwest Provinces. Night after night the whole sky was alight with the burning bungalows, and day after day we had small companies of Europeans passing through our estate with their wives and children, on their way to Agra, where were the nearest troops. Mr Abel White was an obstinate man. He had it in his head that the affair had been exaggerated, and that it would blow over as suddenly as it had sprung up. There he sat on his veranda, drinking whiskey-pegs and smoking cheroots, while the country was in a blaze about him. Of course we stuck by him, I and Dawson, who, with his wife, used to do the book-work and the managing. Well, one fine day the crash came. I had been away on a distant plantation, and was riding slowly home in the evening, when my eye fell upon something all huddled together at the bottom of a steep nullah. I rode down to see what it was, and the cold struck through my heart when I found it was Dawson's wife, all cut into ribbons, and half eaten by jackals and native dogs. A little further up the road Dawson himself was lying on his face, quite dead, with an empty revolver in his hand and four Sepoys lying across each other in front of him. I reined up my horse, wondering which way I should turn, but at that moment I saw thick smoke curling up from Abel White's bungalow and the flames beginning to burst through the roof. I knew then that I could do my employer no good, but would only throw my own life away if I meddled in the matter. From where I stood I could see hundreds of the black fiends, with their red coats still on their backs, dancing and howling round the burning house. Some of them pointed at me, and a couple of bullets sang past my head; so I broke away across the paddy-fields, and found myself late at night safe within the walls at Agra.

As it proved, however, there was no great safety there, either. The whole country was up like a swarm of bees. Wherever the English could collect in little bands they held just the ground that their guns commanded. Everywhere else they were helpless fugitives. It was a fight of the millions against the hundreds; and the cruellest part of it was that these men that we fought against, foot, horse, and gunners, were our own picked troops, whom we had taught and trained, handling our own weapons, and blowing our own bugle-calls. At Agra there were the 3\textsuperscript{rd} Bengal Fusiliers, some Sikhs, two troops of horse, and a battery of artillery. A volunteer corps of clerks and merchants had been formed, and this I joined, wooden leg and all. We went out to meet the rebels at Shahgunge early in July, and we beat them back for a time, but our powder gave out, and we had to fall back upon the city.

Nothing but the worst news came to us from every side,—which is not to be wondered at, for if you look at the map you will see that we were right in the heart of it. Lucknow is rather better than a hundred miles to the east, and Cawnpore about as far to the south. From every point on the compass there was nothing but torture and murder and outrage.

The city of Agra is a great place, swarming with fanatics and fierce devil-worshippers of all sorts. Our handful of men were lost among the narrow, winding streets. Our leader moved across the river, therefore, and took up his position in the old fort at Agra. I don't know if any of you gentlemen have ever read or heard anything of that old fort. It is a very queer place,—the queerest that ever I was in, and I have been in some rum corners, too. First of all, it is enormous in size. I should think that the enclosure must be acres and acres. There is a modern part, which took all our garrison, women, children, stores, and everything else, with plenty of room over. But the modern part is nothing like the size of the old quarter, where nobody goes, and which is given over to the scorpions and the centipedes. It is all full of great deserted halls, and winding passages, and long corridors twisting in and out, so that it is easy enough for folk to get lost in it. For this reason it was seldom that any one went into it, though now and again a party with torches might go exploring.

The river washes along the front of the old fort, and so protects it, but on the sides and behind there are many doors, and these had to be guarded, of course, in the old quarter as well as in that which was actually held by our troops. We were short-handed, with hardly men enough to man the angles of the building and to serve the guns. It was impossible for us, therefore, to station a strong guard at every one of the innumerable gates. What we did was to organise a central guard-house in the middle of the fort, and to leave each gate under the charge of one white man and two or three natives. I was selected to take charge during certain hours of the night of a small isolated door upon the southwest side of the building. Two Sikh troopers were placed under my command, and I was instructed if anything went wrong to fire my musket, when I might rely upon help coming at once from the central guard. As the guard was a good two hundred paces away, however, and as the space between was cut up into a labyrinth of passages and corridors, I had great doubts as to whether they could arrive in time to be of any use in case of an actual attack.

Well, I was pretty proud at having this small command given me, since I was a raw recruit, and a game-legged one at that. For two nights I kept the watch with my Punjaubees. They were tall, fierce-looking chaps, Mahomet Singh and Abdullah Khan by name, both old fighting-men who had borne arms against us at Chilian-wallah. They could talk English pretty well, but I could get little out of them. They preferred to stand together and jabber all night in their queer Sikh lingo. For myself, I used to stand outside the gateway, looking down on the broad, winding river and on the twinkling lights of the great city. The beating of drums, the rattle of tomtoms, and the yells and howls of the rebels, drunk with opium and with bang, were enough to remind us all night of our dangerous neighbours across the stream. Every two hours the officer of the night used to come round to all the posts, to make sure that all was well.

The third night of my watch was dark and dirty, with a small, driving rain. It was dreary work standing in the gateway hour after hour in such weather. I tried again and again to make my Sikhs talk, but without much success. At two in the morning the rounds passed, and broke for a moment the weariness of the night. Finding that my companions would not be led into conversation, I took out my pipe, and laid down my musket to strike the match. In an instant the two Sikhs were upon me. One of them snatched my firelock up and levelled it at my head, while the other held a great knife to my throat and swore between his teeth that he would plunge it into me if I moved a step.

My first thought was that these fellows were in league with the rebels, and that this was the beginning of an assault. If our door were in the hands of the Sepoys the place must fall, and the women and children be treated as they were in Cawnpore. Maybe you gentlemen think that I am just making out a case for myself, but I give you my word that when I thought of that, though I felt the point of the knife at my throat, I opened my mouth with the intention of giving a scream, if it was my last one, which might alarm the main guard. The man who held me seemed to know my thoughts; for, even as I braced myself to it, he whispered, »Don't make a noise. The fort is safe enough. There are no rebel dogs on this side of the river.« There was the ring of truth in what he said, and I knew that if I raised my voice I was a dead man. I could read it in the fellow's brown eyes. I waited, therefore, in silence, to see what it was that they wanted from me.

»Listen to me, Sahib,« said the taller and fiercer of the pair, the one whom they called Abdullah Khan. »You must either be with us now or you must be silenced forever. The thing is too great a one for us to hesitate. Either you are heart and soul with us on your oath on the cross of the Christians, or your body this night shall be thrown into the ditch and we shall pass over to our brothers in the rebel army. There is no middle way. Which is it to be, death or life? We can only give you three minutes to decide, for the time is passing, and all must be done before the rounds come again.«

»How can I decide?« said I. »You have not told me what you want of me. But I tell you now that if it is anything against the safety of the fort I will have no truck with it, so you can drive home your knife and welcome.«

»It is nothing against the fort,« said he. »We only ask you to do that which your countrymen come to this land for. We ask you to be rich. If you will be one of us this night, we will swear to you upon the naked knife, and by the threefold oath which no Sikh was ever known to break, that you shall have your fair share of the loot. A quarter of the treasure shall be yours. We can say no fairer.«

»But what is the treasure, then?« I asked. »I am as ready to be rich as you can be, if you will but show me how it can be done.«

»You will swear, then,« said he, »by the bones of your father, by the honour of your mother, by the cross of your faith, to raise no hand and speak no word against us, either now or afterwards?«

»I will swear it,« I answered, »provided that the fort is not endangered.«

»Then my comrade and I will swear that you shall have a quarter of the treasure which shall be equally divided among the four of us.«

»There are but three,« said I.

»No; Dost Akbar must have his share. We can tell the tale to you while we await them. Do you stand at the gate, Mahomet Singh, and give notice of their coming. The thing stands thus, Sahib, and I tell it to you because I know that an oath is binding upon a Feringhee, and that we may trust you. Had you been a lying Hindoo, though you had sworn by all the gods in their false temples, your blood would have been upon the knife, and your body in the water. But the Sikh knows the Englishman, and the Englishman knows the Sikh. Hearken, then, to what I have to say.

There is a rajah in the northern provinces who has much wealth, though his lands are small. Much has come to him from his father, and more still he has set by himself, for he is of a low nature and hoards his gold rather than spend it. When the troubles broke out he would be friends both with the lion and the tiger,—with the Sepoy and with the Company's Raj. Soon, however, it seemed to him that the white men's day was come, for through all the land he could hear of nothing but of their death and their overthrow. Yet, being a careful man, he made such plans that, come what might, half at least of his treasure should be left to him. That which was in gold and silver he kept by him in the vaults of his palace, but the most precious stones and the choicest pearls that he had he put in an iron box, and sent it by a trusty servant who, under the guise of a merchant, should take it to the fort at Agra, there to lie until the land is at peace. Thus, if the rebels won he would have his money, but if the Company conquered his jewels would be saved to him. Having thus divided his hoard, he threw himself into the cause of the Sepoys, since they were strong upon his borders. By doing this, mark you, Sahib, his property becomes the due of those who have been true to their salt.

This pretended merchant, who travels under the name of Achmet, is now in the city of Agra, and desires to gain his way into the fort. He has with him as travelling-companion my foster-brother Dost Akbar, who knows his secret. Dost Akbar has promised this night to lead him to a side-postern of the fort, and has chosen this one for his purpose. Here he will come presently, and here he will find Mahomet Singh and myself awaiting him. The place is lonely, and none shall know of his coming. The world shall know of the merchant Achmet no more, but the great treasure of the rajah shall be divided among us. What say you to it, Sahib?«

In Worcestershire the life of a man seems a great and a sacred thing; but it is very different when there is fire and blood all round you and you have been used to meeting death at every turn. Whether Achmet the merchant lived or died was a thing as light as air to me, but at the talk about the treasure my heart turned to it, and I thought of what I might do in the old country with it, and how my folk would stare when they saw their ne'er-do-well coming back with his pockets full of gold moidores. I had, therefore, already made up my mind. Abdullah Khan, however, thinking that I hesitated, pressed the matter more closely.

»Consider, Sahib,« said he, »that if this man is taken by the commandant he will be hung or shot, and his jewels taken by the government, so that no man will be a rupee the better for them. Now, since we do the taking of him, why should we not do the rest as well? The jewels will be as well with us as in the Company's coffers. There will be enough to make every one of us rich men and great chiefs. No one can know about the matter, for here we are cut off from all men. What could be better for the purpose? Say again, then, Sahib, whether you are with us, or if we must look upon you as an enemy.«

»I am with you heart and soul,« said I.

»It is well,« he answered, handing me back my firelock. »You see that we trust you, for your word, like ours, is not to be broken. We have now only to wait for my brother and the merchant.«

»Does your brother know, then, of what you will do?« I asked.

»The plan is his. He has devised it. We will go to the gate and share the watch with Mahomet Singh.«

The rain was still falling steadily, for it was just the beginning of the wet season. Brown, heavy clouds were drifting across the sky, and it was hard to see more than a stone-cast. A deep moat lay in front of our door, but the water was in places nearly dried up, and it could easily be crossed. It was strange to me to be standing there with those two wild Punjaubees waiting for the man who was coming to his death.

Suddenly my eye caught the glint of a shaded lantern at the other side of the moat. It vanished among the mound-heaps, and then appeared again coming slowly in our direction.

»Here they are!« I exclaimed.

»You will challenge him, Sahib, as usual,« whispered Abdullah. »Give him no cause for fear. Send us in with him, and we shall do the rest while you stay here on guard. Have the lantern ready to uncover, that we may be sure that it is indeed the man.«

The light had flickered onwards, now stopping and now advancing, until I could see two dark figures upon the other side of the moat. I let them scramble down the sloping bank, splash through the mire, and climb half-way up to the gate, before I challenged them.

»Who goes there?« said I, in a subdued voice.

»Friends,« came the answer. I uncovered my lantern and threw a flood of light upon them. The first was an enormous Sikh, with a black beard which swept nearly down to his cummerbund. Outside of a show I have never seen so tall a man. The other was a little, fat, round fellow, with a great yellow turban, and a bundle in his hand, done up in a shawl. He seemed to be all in a quiver with fear, for his hands twitched as if he had the ague, and his head kept turning to left and right with two bright little twinkling eyes, like a mouse when he ventures out from his hole. It gave me the chills to think of killing him, but I thought of the treasure, and my heart set as hard as a flint within me. When he saw my white face he gave a little chirrup of joy and came running up towards me.

»Your protection, Sahib,« he panted,—»your protection for the unhappy merchant Achmet. I have travelled across Rajpootana that I might seek the shelter of the fort at Agra. I have been robbed and beaten and abused because I have been the friend of the Company. It is a blessed night this when I am once more in safety,—I and my poor possessions.«

»What have you in the bundle?« I asked.

»An iron box,« he answered, »which contains one or two little family matters which are of no value to others, but which I should be sorry to lose. Yet I am not a beggar; and I shall reward you, young Sahib, and your governor also, if he will give me the shelter I ask.«

I could not trust myself to speak longer with the man. The more I looked at his fat, frightened face, the harder did it seem that we should slay him in cold blood. It was best to get it over.

»Take him to the main guard,« said I. The two Sikhs closed in upon him on each side, and the giant walked behind, while they marched in through the dark gateway. Never was a man so compassed round with death. I remained at the gateway with the lantern.

I could hear the measured tramp of their footsteps sounding through the lonely corridors. Suddenly it ceased, and I heard voices, and a scuffle, with the sound of blows. A moment later there came, to my horror, a rush of footsteps coming in my direction, with the loud breathing of a running man. I turned my lantern down the long, straight passage, and there was the fat man, running like the wind, with a smear of blood across his face, and close at his heels, bounding like a tiger, the great black-bearded Sikh, with a knife flashing in his hand. I have never seen a man run so fast as that little merchant. He was gaining on the Sikh, and I could see that if he once passed me and got to the open air he would save himself yet. My heart softened to him, but again the thought of his treasure turned me hard and bitter. I cast my firelock between his legs as he raced past, and he rolled twice over like a shot rabbit. Ere he could stagger to his feet the Sikh was upon him, and buried his knife twice in his side. The man never uttered moan nor moved muscle, but lay were he had fallen. I think myself that he may have broken his neck with the fall. You see, gentlemen, that I am keeping my promise. I am telling you every work of the business just exactly as it happened, whether it is in my favour or not.«

He stopped, and held out his manacled hands for the whiskey-and-water which Holmes had brewed for him. For myself, I confess that I had now conceived the utmost horror of the man, not only for this cold-blooded business in which he had been concerned, but even more for the somewhat flippant and careless way in which he narrated it. Whatever punishment was in store for him, I felt that he might expect no sympathy from me. Sherlock Holmes and Jones sat with their hands upon their knees, deeply interested in the story, but with the same disgust written upon their faces. He may have observed it, for there was a touch of defiance in his voice and manner as he proceeded.

»It was all very bad, no doubt,« said he. »I should like to know how many fellows in my shoes would have refused a share of this loot when they knew that they would have their throats cut for their pains. Besides, it was my life or his when once he was in the fort. If he had got out, the whole business would come to light, and I should have been court-martialled and shot as likely as not; for people were not very lenient at a time like that.«

»Go on with your story,« said Holmes, shortly.

»Well, we carried him in, Abdullah, Akbar, and I. A fine weight he was, too, for all that he was so short. Mahomet Singh was left to guard the door. We took him to a place which the Sikhs had already prepared. It was some distance off, where a winding passage leads to a great empty hall, the brick walls of which were all crumbling to pieces. The earth floor had sunk in at one place, making a natural grave, so we left Achmet the merchant there, having first covered him over with loose bricks. This done, we all went back to the treasure.

It lay where he had dropped it when he was first attacked. The box was the same which now lies open upon your table. A key was hung by a silken cord to that carved handle upon the top. We opened it, and the light of the lantern gleamed upon a collection of gems such as I have read of and thought about when I was a little lad at Pershore. It was blinding to look upon them. When we had feasted our eyes we took them all out and made a list of them. There were one hundred and forty-three diamonds of the first water, including one which has been called, I believe, »the Great Mogul« and is said to be the second largest stone in existence. Then there were ninety-seven very fine emeralds, and one hundred and seventy rubies, some of which, however, were small. There were forty carbuncles, two hundred and ten sapphires, sixty-one agates, and a great quantity of beryls, onyxes, cats'-eyes, turquoises, and other stones, the very names of which I did not know at the time, though I have become more familiar with them since. Besides this, there were nearly three hundred very fine pearls, twelve of which were set in a gold coronet. By the way, these last had been taken out of the chest and were not there when I recovered it.

After we had counted our treasures we put them back into the chest and carried them to the gateway to show them to Mahomet Singh. Then we solemnly renewed our oath to stand by each other and be true to our secret. We agreed to conceal our loot in a safe place until the country should be at peace again, and then to divide it equally among ourselves. There was no use dividing it at present, for if gems of such value were found upon us it would cause suspicion, and there was no privacy in the fort nor any place where we could keep them. We carried the box, therefore, into the same hall where we had buried the body, and there, under certain bricks in the best-preserved wall, we made a hollow and put our treasure. We made careful note of the place, and next day I drew four plans, one for each of us, and put the sign of the four of us at the bottom, for we had sworn that we should each always act for all, so that none might take advantage. That is an oath that I can put my hand to my heart and swear that I have never broken.

Well, there's no use my telling you gentlemen what came of the Indian mutiny. After Wilson took Delhi and Sir Colin relieved Lucknow the back of the business was broken. Fresh troops came pouring in, and Nana Sahib made himself scarce over the frontier. A flying column under Colonel Greathed came round to Agra and cleared the Pandies away from it. Peace seemed to be settling upon the country, and we four were beginning to hope that the time was at hand when we might safely go off with our shares of the plunder. In a moment, however, our hopes were shattered by our being arrested as the murderers of Achmet.

It came about in this way. When the rajah put his jewels into the hands of Achmet he did it because he knew that he was a trusty man. They are suspicious folk in the East, however: so what does this rajah do but take a second even more trusty servant and set him to play the spy upon the first? This second man was ordered never to let Achmet out of his sight, and he followed him like his shadow. He went after him that night and saw him pass through the doorway. Of course he thought he had taken refuge in the fort, and applied for admission there himself next day, but could find no trace of Achmet. This seemed to him so strange that he spoke about it to a sergeant of guides, who brought it to the ears of the commandant. A thorough search was quickly made, and the body was discovered. Thus at the very moment that we thought that all was safe we were all four seized and brought to trial on a charge of murder,—three of us because we had held the gate that night, and the fourth because he was known to have been in the company of the murdered man. Not a word about the jewels came out at the trial, for the rajah had been deposed and driven out of India: so no one had any particular interest in them. The murder, however, was clearly made out, and it was certain that we must all have been concerned in it. The three Sikhs got penal servitude for life, and I was condemned to death, though my sentence was afterwards commuted into the same as the others.

It was rather a queer position that we found ourselves in then. There we were all four tied by the leg and with precious little chance of ever getting out again, while we each held a secret which might have put each of us in a palace if we could only have made use of it. It was enough to make a man eat his heart out to have to stand the kick and the cuff of every petty jack-in-office, to have rice to eat and water to drink, when that gorgeous fortune was ready for him outside, just waiting to be picked up. It might have driven me mad; but I was always a pretty stubborn one, so I just held on and bided my time.

At last it seemed to me to have come. I was changed from Agra to Madras, and from there to Blair Island in the Andamans. There are very few white convicts at this settlement, and, as I had behaved well from the first, I soon found myself a sort of privileged person. I was given a hut in Hope Town, which is a small place on the slopes of Mount Harriet, and I was left pretty much to myself. It is a dreary, fever-stricken place, and all beyond our little clearings was infested with wild cannibal natives, who were ready enough to blow a poisoned dart at us if they saw a chance. There was digging, and ditching, and yam-planting, and a dozen other things to be done, so we were busy enough all day; though in the evening we had a little time to ourselves. Among other things, I learned to dispense drugs for the surgeon, and picked up a smattering of his knowledge. All the time I was on the lookout for a chance of escape; but it is hundreds of miles from any other land, and there is little or no wind in those seas: so it was a terribly difficult job to get away.

The surgeon, Dr Somerton, was a fast, sporting young chap, and the other young officers would meet in his rooms of an evening and play cards. The surgery, where I used to make up my drugs, was next to his sitting-room, with a small window between us. Often, if I felt lonesome, I used to turn out the lamp in the surgery, and then, standing there, I could hear their talk and watch their play. I am fond of a hand at cards myself, and it was almost as good as having one to watch the others. There was Major Sholto, Captain Morstan, and Lieutenant Bromley Brown, who were in command of the native troops, and there was the surgeon himself, and two or three prison-officials, crafty old hands who played a nice sly safe game. A very snug little party they used to make.

Well, there was one thing which very soon struck me, and that was that the soldiers used always to lose and the civilians to win. Mind, I don't say that there was anything unfair, but so it was. These prison-chaps had done little else than play cards ever since they had been at the Andamans, and they knew each other's game to a point, while the others just played to pass the time and threw their cards down anyhow. Night after night the soldiers got up poorer men, and the poorer they got the more keen they were to play. Major Sholto was the hardest hit. He used to pay in notes and gold at first, but soon it came to notes of hand and for big sums. He sometimes would win for a few deals, just to give him heart, and then the luck would set in against him worse than ever. All day he would wander about as black as thunder, and he took to drinking a deal more than was good for him.

One night he lost even more heavily than usual. I was sitting in my hut when he and Captain Morstan came stumbling along on the way to their quarters. They were bosom friends, those two, and never far apart. The major was raving about his losses.

»It's all up, Morstan,« he was saying, as they passed my hut. »I shall have to send in my papers. I am a ruined man.«

»Nonsense, old chap!« said the other, slapping him upon the shoulder. »I've had a nasty facer myself, but—« That was all I could hear, but it was enough to set me thinking.

A couple of days later Major Sholto was strolling on the beach: so I took the chance of speaking to him.

»I wish to have your advice, major,« said I.

»Well, Small, what is it?« he asked, taking his cheroot from his lips.

»I wanted to ask you, sir,« said I, »who is the proper person to whom hidden treasure should be handed over. I know where half a million worth lies, and, as I cannot use it myself, I thought perhaps the best thing that I could do would be to hand it over to the proper authorities, and then perhaps they would get my sentence shortened for me.«

»Half a million, Small?« he gasped, looking hard at me to see if I was in earnest.

»Quite that, sir,—in jewels and pearls. It lies there ready for any one. And the queer thing about it is that the real owner is outlawed and cannot hold property, so that it belongs to the first comer.«

»To government, Small,« he stammered,—»to government.« But he said it in a halting fashion, and I knew in my heart that I had got him.

»You think, then, sir, that I should give the information to the Governor-General?« said I, quietly.

»Well, well, you must not do anything rash, or that you might repent. Let me hear all about it, Small. Give me the facts.«

I told him the whole story, with small changes so that he could not identify the places. When I had finished he stood stock still and full of thought. I could see by the twitch of his lip that there was a struggle going on within him.

»This is a very important matter, Small,« he said, at last. »You must not say a word to any one about it, and I shall see you again soon.«

Two nights later he and his friend Captain Morstan came to my hut in the dead of the night with a lantern.

»I want you just to let Captain Morstan hear that story from your own lips, Small,« said he.

I repeated it as I had told it before.

»It rings true, eh?« said he. »It's good enough to act upon?«

Captain Morstan nodded.

»Look here, Small,« said the major. »We have been talking it over, my friend here and I, and we have come to the conclusion that this secret of yours is hardly a government matter, after all, but is a private concern of your own, which of course you have the power of disposing of as you think best. Now, the question is, what price would you ask for it? We might be inclined to take it up, and at least look into it, if we could agree as to terms.« He tried to speak in a cool, careless way, but his eyes were shining with excitement and greed.

»Why, as to that, gentlemen,« I answered, trying also to be cool, but feeling as excited as he did, »there is only one bargain which a man in my position can make. I shall want you to help me to my freedom, and to help my three companions to theirs. We shall then take you into partnership, and give you a fifth share to divide between you.«

»Hum!« said he. »A fifth share! That is not very tempting.«

»It would come to fifty thousand apiece,« said I.

»But how can we gain your freedom? You know very well that you ask an impossibility.«

»Nothing of the sort,« I answered. »I have thought it all out to the last detail. The only bar to our escape is that we can get no boat fit for the voyage, and no provisions to last us for so long a time. There are plenty of little yachts and yawls at Calcutta or Madras which would serve our turn well. Do you bring one over. We shall engage to get aboard her by night, and if you will drop us on any part of the Indian coast you will have done your part of the bargain.«

»If there were only one,« he said.

»None or all,« I answered. »We have sworn it. The four of us must always act together.«

»You see, Morstan,« said he, »Small is a man of his word. He does not flinch from his friend. I think we may very well trust him.«

»It's a dirty business,« the other answered. »Yet, as you say, the money would save our commissions handsomely.«

»Well, Small,« said the major, »we must, I suppose, try and meet you. We must first, of course, test the truth of your story. Tell me where the box is hid, and I shall get leave of absence and go back to India in the monthly relief-boat to inquire into the affair.«

»Not so fast,« said I, growing colder as he got hot. »I must have the consent of my three comrades. I tell you that it is four or none with us.«

»Nonsense!« he broke in. »What have three black fellows to do with our agreement?«

»Black or blue,« said I, »they are in with me, and we all go together.«

Well, the matter ended by a second meeting, at which Mahomet Singh, Abdullah Khan, and Dost Akbar were all present. We talked the matter over again, and at last we came to an arrangement. We were to provide both the officers with charts of the part of the Agra fort and mark the place in the wall where the treasure was hid. Major Sholto was to go to India to test our story. If he found the box he was to leave it there, to send out a small yacht provisioned for a voyage, which was to lie off Rutland Island, and to which we were to make our way, and finally to return to his duties. Captain Morstan was then to apply for leave of absence, to meet us at Agra, and there we were to have a final division of the treasure, he taking the major's share as well as his own. All this we sealed by the most solemn oaths that the mind could think or the lips utter. I sat up all night with paper and ink, and by the morning I had the two charts all ready, signed with the sign of four,—that is, of Abdullah, Akbar, Mahomet, and myself.

Well, gentlemen, I weary you with my long story, and I know that my friend Mr Jones is impatient to get me safely stowed in chokey. I'll make it as short as I can. The villain Sholto went off to India, but he never came back again. Captain Morstan showed me his name among a list of passengers in one of the mail-boats very shortly afterwards. His uncle had died, leaving him a fortune, and he had left the army, yet he could stoop to treat five men as he had treated us. Morstan went over to Agra shortly afterwards, and found, as we expected, that the treasure was indeed gone. The scoundrel had stolen it all, without carrying out one of the conditions on which we had sold him the secret. From that day I lived only for vengeance. I thought of it by day and I nursed it by night. It became an overpowering, absorbing passion with me. I cared nothing for the law,—nothing for the gallows. To escape, to track down Sholto, to have my hand upon his throat,—that was my one thought. Even the Agra treasure had come to be a smaller thing in my mind than the slaying of Sholto.

Well, I have set my mind on many things in this life, and never one which I did not carry out. But it was weary years before my time came. I have told you that I had picked up something of medicine. One day when Dr Somerton was down with a fever a little Andaman Islander was picked up by a convict-gang in the woods. He was sick to death, and had gone to a lonely place to die. I took him in hand, though he was as venomous as a young snake, and after a couple of months I got him all right and able to walk. He took a kind of fancy to me then, and would hardly go back to his woods, but was always hanging about my hut. I learned a little of his lingo from him, and this made him all the fonder of me.

Tonga—for that was his name—was a fine boatman, and owned a big, roomy canoe of his own. When I found that he was devoted to me and would do anything to serve me, I saw my chance of escape. I talked it over with him. He was to bring his boat round on a certain night to an old wharf which was never guarded, and there he was to pick me up. I gave him directions to have several gourds of water and a lot of yams, cocoa-nuts, and sweet potatoes.

He was stanch and true, was little Tonga. No man ever had a more faithful mate. At the night named he had his boat at the wharf. As it chanced, however, there was one of the convict-guard down there,—a vile Pathan who had never missed a chance of insulting and injuring me. I had always vowed vengeance, and now I had my chance. It was as if fate had placed him in my way that I might pay my debt before I left the island. He stood on the bank with his back to me, and his carbine on his shoulder. I looked about for a stone to beat out his brains with, but none could I see. Then a queer thought came into my head and showed me where I could lay my hand on a weapon. I sat down in the darkness and unstrapped my wooden leg. With three long hops I was on him. He put his carbine to his shoulder, but I struck him full, and knocked the whole front of his skull in. You can see the split in the wood now where I hit him. We both went down together, for I could not keep my balance, but when I got up I found him still lying quiet enough. I made for the boat, and in an hour we were well out at sea. Tonga had brought all his earthly possessions with him, his arms and his gods. Among other things, he had a long bamboo spear, and some Andaman cocoa-nut matting, with which I made a sort of sail. For ten days we were beating about, trusting to luck, and on the eleventh we were picked up by a trader which was going from Singapore to Jiddah with a cargo of Malay pilgrims. They were a rum crowd, and Tonga and I soon managed to settle down among them. They had one very good quality: they let you alone and asked no questions.

Well, if I were to tell you all the adventures that my little chum and I went through, you would not thank me, for I would have you here until the sun was shining. Here and there we drifted about the world, something always turning up to keep us from London. All the time, however, I never lost sight of my purpose. I would dream of Sholto at night. A hundred times I have killed him in my sleep. At last, however, some three or four years ago, we found ourselves in England. I had no great difficulty in finding where Sholto lived, and I set to work to discover whether he had realised the treasure, or if he still had it. I made friends with someone who could help me,—I name no names, for I don't want to get any one else in a hole,—and I soon found that he still had the jewels. Then I tried to get at him in many ways; but he was pretty sly, and had always two prize-fighters, besides his sons and his khitmutgar, on guard over him.

One day, however, I got word that he was dying. I hurried at once to the garden, mad that he should slip out of my clutches like that, and, looking through the window, I saw him lying in his bed, with his sons on each side of him. I'd have come through and taken my chance with the three of them, only even as I looked at him his jaw dropped, and I knew that he was gone. I got into his room that same night, though, and I searched his papers to see if there was any record of where he had hidden our jewels. There was not a line, however: so I came away, bitter and savage as a man could be. Before I left I bethought me that if I ever met my Sikh friends again it would be a satisfaction to know that I had left some mark of our hatred; so I scrawled down the sign of the four of us, as it had been on the chart, and I pinned it on his bosom. It was too much that he should be taken to the grave without some token from the men whom he had robbed and befooled.

We earned a living at this time by my exhibiting poor Tonga at fairs and other such places as the black cannibal. He would eat raw meat and dance his war-dance: so we always had a hatful of pennies after a day's work. I still heard all the news from Pondicherry Lodge, and for some years there was no news to hear, except that they were hunting for the treasure. At last, however, came what we had waited for so long. The treasure had been found. It was up at the top of the house, in Mr Bartholomew Sholto's chemical laboratory. I came at once and had a look at the place, but I could not see how with my wooden leg I was to make my way up to it. I learned, however, about a trap-door in the roof, and also about Mr Sholto's supper-hour. It seemed to me that I could manage the thing easily through Tonga. I brought him out with me with a long rope wound round his waist. He could climb like a cat, and he soon made his way through the roof, but, as ill luck would have it, Bartholomew Sholto was still in the room, to his cost. Tonga thought he had done something very clever in killing him, for when I came up by the rope I found him strutting about as proud as a peacock. Very much surprised was he when I made at him with the rope's end and cursed him for a little blood-thirsty imp. I took the treasure-box and let it down, and then slid down myself, having first left the sign of the four upon the table, to show that the jewels had come back at last to those who had most right to them. Tonga then pulled up the rope, closed the window, and made off the way that he had come.

I don't know that I have anything else to tell you. I had heard a waterman speak of the speed of Smith's launch the \textit{Aurora}, so I thought she would be a handy craft for our escape. I engaged with old Smith, and was to give him a big sum if he got us safe to our ship. He knew, no doubt, that there was some screw loose, but he was not in our secrets. All this is the truth, and if I tell it to you, gentlemen, it is not to amuse you,—for you have not done me a very good turn,—but it is because I believe the best defence I can make is just to hold back nothing, but let all the world know how badly I have myself been served by Major Sholto, and how innocent I am of the death of his son.«

»A very remarkable account,« said Sherlock Holmes. »A fitting wind-up to an extremely interesting case. There is nothing at all new to me in the latter part of your narrative, except that you brought your own rope. That I did not know. By the way, I had hoped that Tonga had lost all his darts; yet he managed to shoot one at us in the boat.«

»He had lost them all, sir, except the one which was in his blow-pipe at the time.«

»Ah, of course,« said Holmes. »I had not thought of that.«

»Is there any other point which you would like to ask about?« asked the convict, affably.

»I think not, thank you,« my companion answered.

»Well, Holmes,« said Athelney Jones, »You are a man to be humoured, and we all know that you are a connoisseur of crime, but duty is duty, and I have gone rather far in doing what you and your friend asked me. I shall feel more at ease when we have our story-teller here safe under lock and key. The cab still waits, and there are two inspectors downstairs. I am much obliged to you both for your assistance. Of course you will be wanted at the trial. Good-night to you.«

»Good-night, gentlemen both,« said Jonathan Small.

»You first, Small,« remarked the wary Jones as they left the room. »I'll take particular care that you don't club me with your wooden leg, whatever you may have done to the gentleman at the Andaman Isles.«

»Well, and there is the end of our little drama,« I remarked, after we had set some time smoking in silence. »I fear that it may be the last investigation in which I shall have the chance of studying your methods. Miss Morstan has done me the honour to accept me as a husband in prospective.«

He gave a most dismal groan. »I feared as much,« said he. »I really cannot congratulate you.«

I was a little hurt. »Have you any reason to be dissatisfied with my choice?« I asked.

»Not at all. I think she is one of the most charming young ladies I ever met, and might have been most useful in such work as we have been doing. She had a decided genius that way: witness the way in which she preserved that Agra plan from all the other papers of her father. But love is an emotional thing, and whatever is emotional is opposed to that true cold reason which I place above all things. I should never marry myself, lest I bias my judgment.«

»I trust,« said I, laughing, »that my judgment may survive the ordeal. But you look weary.«

»Yes, the reaction is already upon me. I shall be as limp as a rag for a week.«

»Strange,« said I, »how terms of what in another man I should call laziness alternate with your fits of splendid energy and vigour.«

»Yes,« he answered, »there are in me the makings of a very fine loafer and also of a pretty spry sort of fellow. I often think of those lines of old Goethe,—
\begin{german}
\begin{verse}
Schade dass die Natur nur \textit{einen} Mensch aus Dir schuf,\\
Denn zum würdigen Mann war und zum Schelmen der Stoff.
\end{verse}
\end{german}«

»By the way, \textit{à propos} of this Norwood business, you see that they had, as I surmised, a confederate in the house, who could be none other than Lal Rao, the butler: so Jones actually has the undivided honour of having caught one fish in his great haul.«

»The division seems rather unfair,« I remarked. »You have done all the work in this business. I get a wife out of it, Jones gets the credit, pray what remains for you?«

»For me,« said Sherlock Holmes, »there still remains the cocaine-bottle.« And he stretched his long white hand up for it.