%!TeX root=../signtop.tex
\chapter{The Statement of the Case}
\lettrine[lines=4]{M}{iss} Morstan entered the room with a firm step and an outward composure of manner. She was a blonde young lady, small, dainty, well gloved, and dressed in the most perfect taste. There was, however, a plainness and simplicity about her costume which bore with it a suggestion of limited means. The dress was a sombre greyish beige, untrimmed and unbraided, and she wore a small turban of the same dull hue, relieved only by a suspicion of white feather in the side. Her face had neither regularity of feature nor beauty of complexion, but her expression was sweet and amiable, and her large blue eyes were singularly spiritual and sympathetic. In an experience of women which extends over many nations and three separate continents, I have never looked upon a face which gave a clearer promise of a refined and sensitive nature. I could not but observe that as she took the seat which Sherlock Holmes placed for her, her lip trembled, her hand quivered, and she showed every sign of intense inward agitation.

»I have come to you, Mr Holmes,« she said, »because you once enabled my employer, Mrs Cecil Forrester, to unravel a little domestic complication. She was much impressed by your kindness and skill.«

»Mrs Cecil Forrester,« he repeated thoughtfully. »I believe that I was of some slight service to her. The case, however, as I remember it, was a very simple one.«

»She did not think so. But at least you cannot say the same of mine. I can hardly imagine anything more strange, more utterly inexplicable, than the situation in which I find myself.«

Holmes rubbed his hands, and his eyes glistened. He leaned forward in his chair with an expression of extraordinary concentration upon his clear-cut, hawklike features. »State your case,« said he, in brisk, business tones.

I felt that my position was an embarrassing one. »You will, I am sure, excuse me,« I said, rising from my chair.

To my surprise, the young lady held up her gloved hand to detain me. »If your friend,« she said, »would be good enough to stop, he might be of inestimable service to me.«

I relapsed into my chair.

»Briefly,« she continued, »the facts are these. My father was an officer in an Indian regiment who sent me home when I was quite a child. My mother was dead, and I had no relative in England. I was placed, however, in a comfortable boarding establishment at Edinburgh, and there I remained until I was seventeen years of age. In the year 1878 my father, who was senior captain of his regiment, obtained twelve months' leave and came home. He telegraphed to me from London that he had arrived all safe, and directed me to come down at once, giving the Langham Hotel as his address. His message, as I remember, was full of kindness and love. On reaching London I drove to the Langham, and was informed that Captain Morstan was staying there, but that he had gone out the night before and had not yet returned. I waited all day without news of him. That night, on the advice of the manager of the hotel, I communicated with the police, and next morning we advertised in all the papers. Our inquiries led to no result; and from that day to this no word has ever been heard of my unfortunate father. He came home with his heart full of hope, to find some peace, some comfort, and instead\longdash« She put her hand to her throat, and a choking sob cut short the sentence.

»The date?« asked Holmes, opening his note-book.

»He disappeared upon the 3\textsuperscript{rd} of December, 1878,—nearly ten years ago.«

»His luggage?«

»Remained at the hotel. There was nothing in it to suggest a clue,—some clothes, some books, and a considerable number of curiosities from the Andaman Islands. He had been one of the officers in charge of the convict-guard there.«

»Had he any friends in town?«

»Only one that we know of,—Major Sholto, of his own regiment, the 34\textsuperscript{th} Bombay Infantry. The major had retired some little time before, and lived at Upper Norwood. We communicated with him, of course, but he did not even know that his brother officer was in England.«

»A singular case,« remarked Holmes.

»I have not yet described to you the most singular part. About six years ago—to be exact, upon the 4\textsuperscript{th} of May, 1882—an advertisement appeared in the \textit{Times} asking for the address of Miss Mary Morstan and stating that it would be to her advantage to come forward. There was no name or address appended. I had at that time just entered the family of Mrs Cecil Forrester in the capacity of governess. By her advice I published my address in the advertisement column. The same day there arrived through the post a small card-board box addressed to me, which I found to contain a very large and lustrous pearl. No word of writing was enclosed. Since then every year upon the same date there has always appeared a similar box, containing a similar pearl, without any clue as to the sender. They have been pronounced by an expert to be of a rare variety and of considerable value. You can see for yourselves that they are very handsome.« She opened a flat box as she spoke, and showed me six of the finest pearls that I had ever seen.

»Your statement is most interesting,« said Sherlock Holmes. »Has anything else occurred to you?«

»Yes, and no later than to-day. That is why I have come to you. This morning I received this letter, which you will perhaps read for yourself.«

»Thank you,« said Holmes. »The envelope too, please. Postmark, London, S.W. Date, July 7. Hum! Man's thumb-mark on corner,—probably postman. Best quality paper. Envelopes at sixpence a packet. Particular man in his stationery. No address. »Be at the third pillar from the left outside the Lyceum Theatre to-night at seven o'clock. If you are distrustful, bring two friends. You are a wronged woman, and shall have justice. Do not bring police. If you do, all will be in vain. Your unknown friend.« Well, really, this is a very pretty little mystery. What do you intend to do, Miss Morstan?«

»That is exactly what I want to ask you.«

»Then we shall most certainly go. You and I and—yes, why, Dr Watson is the very man. Your correspondent says two friends. He and I have worked together before.«

»But would he come?« she asked, with something appealing in her voice and expression.

»I should be proud and happy,« said I, fervently, »if I can be of any service.«

»You are both very kind,« she answered. »I have led a retired life, and have no friends whom I could appeal to. If I am here at six it will do, I suppose?«

»You must not be later,« said Holmes. »There is one other point, however. Is this handwriting the same as that upon the pearl-box addresses?«

»I have them here,« she answered, producing half a dozen pieces of paper.

»You are certainly a model client. You have the correct intuition. Let us see, now.« He spread out the papers upon the table, and gave little darting glances from one to the other. »They are disguised hands, except the letter,« he said, presently, »but there can be no question as to the authorship. See how the irrepressible Greek \textit{e} will break out, and see the twirl of the final \textit{s}. They are undoubtedly by the same person. I should not like to suggest false hopes, Miss Morstan, but is there any resemblance between this hand and that of your father?«

»Nothing could be more unlike.«

»I expected to hear you say so. We shall look out for you, then, at six. Pray allow me to keep the papers. I may look into the matter before then. It is only half-past three. \textit{Au revoir,} then.«

»\textit{Au revoir,}« said our visitor, and, with a bright, kindly glance from one to the other of us, she replaced her pearl-box in her bosom and hurried away. Standing at the window, I watched her walking briskly down the street, until the grey turban and white feather were but a speck in the sombre crowd.

»What a very attractive woman!« I exclaimed, turning to my companion.

He had lit his pipe again, and was leaning back with drooping eyelids. »Is she?« he said, languidly. »I did not observe.«

»You really are an automaton,—a calculating-machine!« I cried. »There is something positively inhuman in you at times.«

He smiled gently. »It is of the first importance,« he said, »not to allow your judgment to be biased by personal qualities. A client is to me a mere unit,—a factor in a problem. The emotional qualities are antagonistic to clear reasoning. I assure you that the most winning woman I ever knew was hanged for poisoning three little children for their insurance-money, and the most repellent man of my acquaintance is a philanthropist who has spent nearly a quarter of a million upon the London poor.«

»In this case, however\longdash«

»I never make exceptions. An exception disproves the rule. Have you ever had occasion to study character in handwriting? What do you make of this fellow's scribble?«

»It is legible and regular,« I answered. »A man of business habits and some force of character.«

Holmes shook his head. »Look at his long letters,« he said. »They hardly rise above the common herd. That \textit{d} might be an \textit{a}, and that \textit{l} an \textit{e}. Men of character always differentiate their long letters, however illegibly they may write. There is vacillation in his \textit{k}'s and self-esteem in his capitals. I am going out now. I have some few references to make. Let me recommend this book,—one of the most remarkable ever penned. It is Winwood Reade's \textit{Martyrdom of Man}. I shall be back in an hour.«

I sat in the window with the volume in my hand, but my thoughts were far from the daring speculations of the writer. My mind ran upon our late visitor,—her smiles, the deep rich tones of her voice, the strange mystery which overhung her life. If she were seventeen at the time of her father's disappearance she must be seven-and-twenty now,—a sweet age, when youth has lost its self-consciousness and become a little sobered by experience. So I sat and mused, until such dangerous thoughts came into my head that I hurried away to my desk and plunged furiously into the latest treatise upon pathology. What was I, an army surgeon with a weak leg and a weaker banking-account, that I should dare to think of such things? She was a unit, a factor,—nothing more. If my future were black, it was better surely to face it like a man than to attempt to brighten it by mere will-o'-the-wisps of the imagination.