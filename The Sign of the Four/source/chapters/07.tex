%!TeX root=../signtop.tex
\chapter{The Episode of the Barrel}
\lettrine[lines=4]{T}{he} police had brought a cab with them, and in this I escorted Miss Morstan back to her home. After the angelic fashion of women, she had borne trouble with a calm face as long as there was some one weaker than herself to support, and I had found her bright and placid by the side of the frightened housekeeper. In the cab, however, she first turned faint, and then burst into a passion of weeping,—so sorely had she been tried by the adventures of the night. She has told me since that she thought me cold and distant upon that journey. She little guessed the struggle within my breast, or the effort of self-restraint which held me back. My sympathies and my love went out to her, even as my hand had in the garden. I felt that years of the conventionalities of life could not teach me to know her sweet, brave nature as had this one day of strange experiences. Yet there were two thoughts which sealed the words of affection upon my lips. She was weak and helpless, shaken in mind and nerve. It was to take her at a disadvantage to obtrude love upon her at such a time. Worse still, she was rich. If Holmes's researches were successful, she would be an heiress. Was it fair, was it honourable, that a half-pay surgeon should take such advantage of an intimacy which chance had brought about? Might she not look upon me as a mere vulgar fortune-seeker? I could not bear to risk that such a thought should cross her mind. This Agra treasure intervened like an impassable barrier between us.

It was nearly two o'clock when we reached Mrs Cecil Forrester's. The servants had retired hours ago, but Mrs Forrester had been so interested by the strange message which Miss Morstan had received that she had sat up in the hope of her return. She opened the door herself, a middle-aged, graceful woman, and it gave me joy to see how tenderly her arm stole round the other's waist and how motherly was the voice in which she greeted her. She was clearly no mere paid dependant, but an honoured friend. I was introduced, and Mrs Forrester earnestly begged me to step in and tell her our adventures. I explained, however, the importance of my errand, and promised faithfully to call and report any progress which we might make with the case. As we drove away I stole a glance back, and I still seem to see that little group on the step, the two graceful, clinging figures, the half-opened door, the hall-light shining through stained glass, the barometer, and the bright stair-rods. It was soothing to catch even that passing glimpse of a tranquil English home in the midst of the wild, dark business which had absorbed us.

And the more I thought of what had happened, the wilder and darker it grew. I reviewed the whole extraordinary sequence of events as I rattled on through the silent gas-lit streets. There was the original problem: that at least was pretty clear now. The death of Captain Morstan, the sending of the pearls, the advertisement, the letter,—we had had light upon all those events. They had only led us, however, to a deeper and far more tragic mystery. The Indian treasure, the curious plan found among Morstan's baggage, the strange scene at Major Sholto's death, the rediscovery of the treasure immediately followed by the murder of the discoverer, the very singular accompaniments to the crime, the footsteps, the remarkable weapons, the words upon the card, corresponding with those upon Captain Morstan's chart,—here was indeed a labyrinth in which a man less singularly endowed than my fellow-lodger might well despair of ever finding the clue.

Pinchin Lane was a row of shabby two-storied brick houses in the lower quarter of Lambeth. I had to knock for some time at No. 3 before I could make my impression. At last, however, there was the glint of a candle behind the blind, and a face looked out at the upper window.

»Go on, you drunken vagabone,« said the face. »If you kick up any more row I'll open the kennels and let out forty-three dogs upon you.«

»If you'll let one out it's just what I have come for,« said I.

»Go on!« yelled the voice. »So help me gracious, I have a wiper in the bag, an' I'll drop it on your 'ead if you don't hook it.«

»But I want a dog,« I cried.

»I won't be argued with!« shouted Mr Sherman. »Now stand clear, for when I say »three,« down goes the wiper.«

»Mr Sherlock Holmes\longdash« I began, but the words had a most magical effect, for the window instantly slammed down, and within a minute the door was unbarred and open. Mr Sherman was a lanky, lean old man, with stooping shoulders, a stringy neck, and blue-tinted glasses.

»A friend of Mr Sherlock is always welcome,« said he. »Step in, sir. Keep clear of the badger; for he bites. Ah, naughty, naughty, would you take a nip at the gentleman?« This to a stoat which thrust its wicked head and red eyes between the bars of its cage. »Don't mind that, sir: it's only a slow-worm. It hain't got no fangs, so I gives it the run o' the room, for it keeps the beetles down. You must not mind my bein' just a little short wi' you at first, for I'm guyed at by the children, and there's many a one just comes down this lane to knock me up. What was it that Mr Sherlock Holmes wanted, sir?«

»He wanted a dog of yours.«

»Ah! that would be Toby.«

»Yes, Toby was the name.«

»Toby lives at No. 7 on the left here.« He moved slowly forward with his candle among the queer animal family which he had gathered round him. In the uncertain, shadowy light I could see dimly that there were glancing, glimmering eyes peeping down at us from every cranny and corner. Even the rafters above our heads were lined by solemn fowls, who lazily shifted their weight from one leg to the other as our voices disturbed their slumbers.

Toby proved to be an ugly, long-haired, lop-eared creature, half spaniel and half lurcher, brown-and-white in colour, with a very clumsy waddling gait. It accepted after some hesitation a lump of sugar which the old naturalist handed to me, and, having thus sealed an alliance, it followed me to the cab, and made no difficulties about accompanying me. It had just struck three on the Palace clock when I found myself back once more at Pondicherry Lodge. The ex-prize-fighter McMurdo had, I found, been arrested as an accessory, and both he and Mr Sholto had been marched off to the station. Two constables guarded the narrow gate, but they allowed me to pass with the dog on my mentioning the detective's name.

Holmes was standing on the door-step, with his hands in his pockets, smoking his pipe.

»Ah, you have him there!« said he. »Good dog, then! Atheney Jones has gone. We have had an immense display of energy since you left. He has arrested not only friend Thaddeus, but the gatekeeper, the housekeeper, and the Indian servant. We have the place to ourselves, but for a sergeant upstairs. Leave the dog here, and come up.«

We tied Toby to the hall table, and re-ascended the stairs. The room was as he had left it, save that a sheet had been draped over the central figure. A weary-looking police-sergeant reclined in the corner.

»Lend me your bull's-eye, sergeant,« said my companion. »Now tie this bit of card round my neck, so as to hang it in front of me. Thank you. Now I must kick off my boots and stockings.—Just you carry them down with you, Watson. I am going to do a little climbing. And dip my handkerchief into the creasote. That will do. Now come up into the garret with me for a moment.«

We clambered up through the hole. Holmes turned his light once more upon the footsteps in the dust.

»I wish you particularly to notice these footmarks,« he said. »Do you observe anything noteworthy about them?«

»They belong,« I said, »to a child or a small woman.«

»Apart from their size, though. Is there nothing else?«

»They appear to be much as other footmarks.«

»Not at all. Look here! This is the print of a right foot in the dust. Now I make one with my naked foot beside it. What is the chief difference?«

»Your toes are all cramped together. The other print has each toe distinctly divided.«

»Quite so. That is the point. Bear that in mind. Now, would you kindly step over to that flap-window and smell the edge of the wood-work? I shall stay here, as I have this handkerchief in my hand.«

I did as he directed, and was instantly conscious of a strong tarry smell.

»That is where he put his foot in getting out. If \textit{you} can trace him, I should think that Toby will have no difficulty. Now run downstairs, loose the dog, and look out for Blondin.«

By the time that I got out into the grounds Sherlock Holmes was on the roof, and I could see him like an enormous glow-worm crawling very slowly along the ridge. I lost sight of him behind a stack of chimneys, but he presently reappeared, and then vanished once more upon the opposite side. When I made my way round there I found him seated at one of the corner eaves.

»That you, Watson?« he cried.

»Yes.«

»This is the place. What is that black thing down there?«

»A water-barrel.«

»Top on it?«

»Yes.«

»No sign of a ladder?«

»No.«

»Confound the fellow! It's a most break-neck place. I ought to be able to come down where he could climb up. The water-pipe feels pretty firm. Here goes, anyhow.«

There was a scuffling of feet, and the lantern began to come steadily down the side of the wall. Then with a light spring he came on to the barrel, and from there to the earth.

»It was easy to follow him,« he said, drawing on his stockings and boots. »Tiles were loosened the whole way along, and in his hurry he had dropped this. It confirms my diagnosis, as you doctors express it.«

The object which he held up to me was a small pocket or pouch woven out of coloured grasses and with a few tawdry beads strung round it. In shape and size it was not unlike a cigarette-case. Inside were half a dozen spines of dark wood, sharp at one end and rounded at the other, like that which had struck Bartholomew Sholto.

»They are hellish things,« said he. »Look out that you don't prick yourself. I'm delighted to have them, for the chances are that they are all he has. There is the less fear of you or me finding one in our skin before long. I would sooner face a Martini bullet, myself. Are you game for a six-mile trudge, Watson?«

»Certainly,« I answered.

»Your leg will stand it?«

»Oh, yes.«

»Here you are, doggy! Good old Toby! Smell it, Toby, smell it!« He pushed the creasote handkerchief under the dog's nose, while the creature stood with its fluffy legs separated, and with a most comical cock to its head, like a connoisseur sniffing the \textit{bouquet} of a famous vintage. Holmes then threw the handkerchief to a distance, fastened a stout cord to the mongrel's collar, and led him to the foot of the water-barrel. The creature instantly broke into a succession of high, tremulous yelps, and, with his nose on the ground, and his tail in the air, pattered off upon the trail at a pace which strained his leash and kept us at the top of our speed.

The east had been gradually whitening, and we could now see some distance in the cold grey light. The square, massive house, with its black, empty windows and high, bare walls, towered up, sad and forlorn, behind us. Our course led right across the grounds, in and out among the trenches and pits with which they were scarred and intersected. The whole place, with its scattered dirt-heaps and ill-grown shrubs, had a blighted, ill-omened look which harmonized with the black tragedy which hung over it.

On reaching the boundary wall Toby ran along, whining eagerly, underneath its shadow, and stopped finally in a corner screened by a young beech. Where the two walls joined, several bricks had been loosened, and the crevices left were worn down and rounded upon the lower side, as though they had frequently been used as a ladder. Holmes clambered up, and, taking the dog from me, he dropped it over upon the other side.

»There's the print of wooden-leg's hand,« he remarked, as I mounted up beside him. »You see the slight smudge of blood upon the white plaster. What a lucky thing it is that we have had no very heavy rain since yesterday! The scent will lie upon the road in spite of their eight-and-twenty hours' start.«

I confess that I had my doubts myself when I reflected upon the great traffic which had passed along the London road in the interval. My fears were soon appeased, however. Toby never hesitated or swerved, but waddled on in his peculiar rolling fashion. Clearly, the pungent smell of the creasote rose high above all other contending scents.

»Do not imagine,« said Holmes, »that I depend for my success in this case upon the mere chance of one of these fellows having put his foot in the chemical. I have knowledge now which would enable me to trace them in many different ways. This, however, is the readiest and, since fortune has put it into our hands, I should be culpable if I neglected it. It has, however, prevented the case from becoming the pretty little intellectual problem which it at one time promised to be. There might have been some credit to be gained out of it, but for this too palpable clue.«

»There is credit, and to spare,« said I. »I assure you, Holmes, that I marvel at the means by which you obtain your results in this case, even more than I did in the Jefferson Hope Murder. The thing seems to me to be deeper and more inexplicable. How, for example, could you describe with such confidence the wooden-legged man?«

»Pshaw, my dear boy! it was simplicity itself. I don't wish to be theatrical. It is all patent and above-board. Two officers who are in command of a convict-guard learn an important secret as to buried treasure. A map is drawn for them by an Englishman named Jonathan Small. You remember that we saw the name upon the chart in Captain Morstan's possession. He had signed it in behalf of himself and his associates,—the sign of the four, as he somewhat dramatically called it. Aided by this chart, the officers—or one of them—gets the treasure and brings it to England, leaving, we will suppose, some condition under which he received it unfulfilled. Now, then, why did not Jonathan Small get the treasure himself? The answer is obvious. The chart is dated at a time when Morstan was brought into close association with convicts. Jonathan Small did not get the treasure because he and his associates were themselves convicts and could not get away.«

»But that is mere speculation,« said I.

»It is more than that. It is the only hypothesis which covers the facts. Let us see how it fits in with the sequel. Major Sholto remains at peace for some years, happy in the possession of his treasure. Then he receives a letter from India which gives him a great fright. What was that?«

»A letter to say that the men whom he had wronged had been set free.«

»Or had escaped. That is much more likely, for he would have known what their term of imprisonment was. It would not have been a surprise to him. What does he do then? He guards himself against a wooden-legged man,—a white man, mark you, for he mistakes a white tradesman for him, and actually fires a pistol at him. Now, only one white man's name is on the chart. The others are Hindoos or Mohammedans. There is no other white man. Therefore we may say with confidence that the wooden-legged man is identical with Jonathan Small. Does the reasoning strike you as being faulty?«

»No: it is clear and concise.«

»Well, now, let us put ourselves in the place of Jonathan Small. Let us look at it from his point of view. He comes to England with the double idea of regaining what he would consider to be his rights and of having his revenge upon the man who had wronged him. He found out where Sholto lived, and very possibly he established communications with some one inside the house. There is this butler, Lal Rao, whom we have not seen. Mrs Bernstone gives him far from a good character. Small could not find out, however, where the treasure was hid, for no one ever knew, save the major and one faithful servant who had died. Suddenly Small learns that the major is on his death-bed. In a frenzy lest the secret of the treasure die with him, he runs the gauntlet of the guards, makes his way to the dying man's window, and is only deterred from entering by the presence of his two sons. Mad with hate, however, against the dead man, he enters the room that night, searches his private papers in the hope of discovering some memorandum relating to the treasure, and finally leaves a memento of his visit in the short inscription upon the card. He had doubtless planned beforehand that should he slay the major he would leave some such record upon the body as a sign that it was not a common murder, but, from the point of view of the four associates, something in the nature of an act of justice. Whimsical and bizarre conceits of this kind are common enough in the annals of crime, and usually afford valuable indications as to the criminal. Do you follow all this?«

»Very clearly.«

»Now, what could Jonathan Small do? He could only continue to keep a secret watch upon the efforts made to find the treasure. Possibly he leaves England and only comes back at intervals. Then comes the discovery of the garret, and he is instantly informed of it. We again trace the presence of some confederate in the household. Jonathan, with his wooden leg, is utterly unable to reach the lofty room of Bartholomew Sholto. He takes with him, however, a rather curious associate, who gets over this difficulty, but dips his naked foot into creasote, whence comes Toby, and a six-mile limp for a half-pay officer with a damaged tendo Achillis.«

»But it was the associate, and not Jonathan, who committed the crime.«

»Quite so. And rather to Jonathan's disgust, to judge by the way he stamped about when he got into the room. He bore no grudge against Bartholomew Sholto, and would have preferred if he could have been simply bound and gagged. He did not wish to put his head in a halter. There was no help for it, however: the savage instincts of his companion had broken out, and the poison had done its work: so Jonathan Small left his record, lowered the treasure-box to the ground, and followed it himself. That was the train of events as far as I can decipher them. Of course as to his personal appearance he must be middle-aged, and must be sunburned after serving his time in such an oven as the Andamans. His height is readily calculated from the length of his stride, and we know that he was bearded. His hairiness was the one point which impressed itself upon Thaddeus Sholto when he saw him at the window. I don't know that there is anything else.«

»The associate?«

»Ah, well, there is no great mystery in that. But you will know all about it soon enough. How sweet the morning air is! See how that one little cloud floats like a pink feather from some gigantic flamingo. Now the red rim of the sun pushes itself over the London cloud-bank. It shines on a good many folk, but on none, I dare bet, who are on a stranger errand than you and I. How small we feel with our petty ambitions and strivings in the presence of the great elemental forces of nature! Are you well up in your Jean Paul?«

»Fairly so. I worked back to him through Carlyle.«

»That was like following the brook to the parent lake. He makes one curious but profound remark. It is that the chief proof of man's real greatness lies in his perception of his own smallness. It argues, you see, a power of comparison and of appreciation which is in itself a proof of nobility. There is much food for thought in Richter. You have not a pistol, have you?«

»I have my stick.«

»It is just possible that we may need something of the sort if we get to their lair. Jonathan I shall leave to you, but if the other turns nasty I shall shoot him dead.« He took out his revolver as he spoke, and, having loaded two of the chambers, he put it back into the right-hand pocket of his jacket.

We had during this time been following the guidance of Toby down the half-rural villa-lined roads which lead to the metropolis. Now, however, we were beginning to come among continuous streets, where labourers and dockmen were already astir, and slatternly women were taking down shutters and brushing door-steps. At the square-topped corner public houses business was just beginning, and rough-looking men were emerging, rubbing their sleeves across their beards after their morning wet. Strange dogs sauntered up and stared wonderingly at us as we passed, but our inimitable Toby looked neither to the right nor to the left, but trotted onwards with his nose to the ground and an occasional eager whine which spoke of a hot scent.

We had traversed Streatham, Brixton, Camberwell, and now found ourselves in Kennington Lane, having borne away through the side-streets to the east of the Oval. The men whom we pursued seemed to have taken a curiously zigzag road, with the idea probably of escaping observation. They had never kept to the main road if a parallel side-street would serve their turn. At the foot of Kennington Lane they had edged away to the left through Bond Street and Miles Street. Where the latter street turns into Knight's Place, Toby ceased to advance, but began to run backwards and forwards with one ear cocked and the other drooping, the very picture of canine indecision. Then he waddled round in circles, looking up to us from time to time, as if to ask for sympathy in his embarrassment.

»What the deuce is the matter with the dog?« growled Holmes. »They surely would not take a cab, or go off in a balloon.«

»Perhaps they stood here for some time,« I suggested.

»Ah! it's all right. He's off again,« said my companion, in a tone of relief.

He was indeed off, for after sniffing round again he suddenly made up his mind, and darted away with an energy and determination such as he had not yet shown. The scent appeared to be much hotter than before, for he had not even to put his nose on the ground, but tugged at his leash and tried to break into a run. I could see by the gleam in Holmes's eyes that he thought we were nearing the end of our journey.

Our course now ran down Nine Elms until we came to Broderick and Nelson's large timber-yard, just past the White Eagle tavern. Here the dog, frantic with excitement, turned down through the side-gate into the enclosure, where the sawyers were already at work. On the dog raced through sawdust and shavings, down an alley, round a passage, between two wood-piles, and finally, with a triumphant yelp, sprang upon a large barrel which still stood upon the hand-trolley on which it had been brought. With lolling tongue and blinking eyes, Toby stood upon the cask, looking from one to the other of us for some sign of appreciation. The staves of the barrel and the wheels of the trolley were smeared with a dark liquid, and the whole air was heavy with the smell of creasote.

Sherlock Holmes and I looked blankly at each other, and then burst simultaneously into an uncontrollable fit of laughter.