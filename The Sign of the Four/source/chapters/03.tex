%!TeX root=../signtop.tex
\chapter{In Quest of a Solution}
\lettrine[lines=4]{I}{t} was half-past five before Holmes returned. He was bright, eager, and in excellent spirits,—a mood which in his case alternated with fits of the blackest depression.

\zz
»There is no great mystery in this matter,« he said, taking the cup of tea which I had poured out for him. »The facts appear to admit of only one explanation.«

»What! you have solved it already?«

»Well, that would be too much to say. I have discovered a suggestive fact, that is all. It is, however, \textit{very} suggestive. The details are still to be added. I have just found, on consulting the back files of the \textit{Times}, that Major Sholto, of Upper Norwood, late of the 34\textsuperscript{th} Bombay Infantry, died upon the 28\textsuperscript{th} of April, 1882.«

»I may be very obtuse, Holmes, but I fail to see what this suggests.«

»No? You surprise me. Look at it in this way, then. Captain Morstan disappears. The only person in London whom he could have visited is Major Sholto. Major Sholto denies having heard that he was in London. Four years later Sholto dies. \textit{Within a week of his death} Captain Morstan's daughter receives a valuable present, which is repeated from year to year, and now culminates in a letter which describes her as a wronged woman. What wrong can it refer to except this deprivation of her father? And why should the presents begin immediately after Sholto's death, unless it is that Sholto's heir knows something of the mystery and desires to make compensation? Have you any alternative theory which will meet the facts?«

»But what a strange compensation! And how strangely made! Why, too, should he write a letter now, rather than six years ago? Again, the letter speaks of giving her justice. What justice can she have? It is too much to suppose that her father is still alive. There is no other injustice in her case that you know of.«

»There are difficulties; there are certainly difficulties,« said Sherlock Holmes, pensively. »But our expedition of to-night will solve them all. Ah, here is a four-wheeler, and Miss Morstan is inside. Are you all ready? Then we had better go down, for it is a little past the hour.«

I picked up my hat and my heaviest stick, but I observed that Holmes took his revolver from his drawer and slipped it into his pocket. It was clear that he thought that our night's work might be a serious one.

Miss Morstan was muffled in a dark cloak, and her sensitive face was composed, but pale. She must have been more than woman if she did not feel some uneasiness at the strange enterprise upon which we were embarking, yet her self-control was perfect, and she readily answered the few additional questions which Sherlock Holmes put to her.

»Major Sholto was a very particular friend of papa's,« she said. »His letters were full of allusions to the major. He and papa were in command of the troops at the Andaman Islands, so they were thrown a great deal together. By the way, a curious paper was found in papa's desk which no one could understand. I don't suppose that it is of the slightest importance, but I thought you might care to see it, so I brought it with me. It is here.«

Holmes unfolded the paper carefully and smoothed it out upon his knee. He then very methodically examined it all over with his double lens.

»It is paper of native Indian manufacture,« he remarked. »It has at some time been pinned to a board. The diagram upon it appears to be a plan of part of a large building with numerous halls, corridors, and passages. At one point is a small cross done in red ink, and above it is »3.37 from left,« in faded pencil-writing. In the left-hand corner is a curious hieroglyphic like four crosses in a line with their arms touching. Beside it is written, in very rough and coarse characters, »The sign of the four,—Jonathan Small, Mahomet Singh, Abdullah Khan, Dost Akbar.« No, I confess that I do not see how this bears upon the matter. Yet it is evidently a document of importance. It has been kept carefully in a pocket-book; for the one side is as clean as the other.«

»It was in his pocket-book that we found it.«

»Preserve it carefully, then, Miss Morstan, for it may prove to be of use to us. I begin to suspect that this matter may turn out to be much deeper and more subtle than I at first supposed. I must reconsider my ideas.« He leaned back in the cab, and I could see by his drawn brow and his vacant eye that he was thinking intently. Miss Morstan and I chatted in an undertone about our present expedition and its possible outcome, but our companion maintained his impenetrable reserve until the end of our journey.

It was a September evening, and not yet seven o'clock, but the day had been a dreary one, and a dense drizzly fog lay low upon the great city. Mud-coloured clouds drooped sadly over the muddy streets. Down the Strand the lamps were but misty splotches of diffused light which threw a feeble circular glimmer upon the slimy pavement. The yellow glare from the shop-windows streamed out into the steamy, vaporous air, and threw a murky, shifting radiance across the crowded thoroughfare. There was, to my mind, something eerie and ghost-like in the endless procession of faces which flitted across these narrow bars of light,—sad faces and glad, haggard and merry. Like all human kind, they flitted from the gloom into the light, and so back into the gloom once more. I am not subject to impressions, but the dull, heavy evening, with the strange business upon which we were engaged, combined to make me nervous and depressed. I could see from Miss Morstan's manner that she was suffering from the same feeling. Holmes alone could rise superior to petty influences. He held his open note-book upon his knee, and from time to time he jotted down figures and memoranda in the light of his pocket-lantern.

At the Lyceum Theatre the crowds were already thick at the side-entrances. In front a continuous stream of hansoms and four-wheelers were rattling up, discharging their cargoes of shirt-fronted men and beshawled, bediamonded women. We had hardly reached the third pillar, which was our rendezvous, before a small, dark, brisk man in the dress of a coachman accosted us.

»Are you the parties who come with Miss Morstan?« he asked.

»I am Miss Morstan, and these two gentlemen are my friends,« said she.

He bent a pair of wonderfully penetrating and questioning eyes upon us. »You will excuse me, miss,« he said with a certain dogged manner, »but I was to ask you to give me your word that neither of your companions is a police-officer.«

»I give you my word on that,« she answered.

He gave a shrill whistle, on which a street Arab led across a four-wheeler and opened the door. The man who had addressed us mounted to the box, while we took our places inside. We had hardly done so before the driver whipped up his horse, and we plunged away at a furious pace through the foggy streets.

The situation was a curious one. We were driving to an unknown place, on an unknown errand. Yet our invitation was either a complete hoax,—which was an inconceivable hypothesis,—or else we had good reason to think that important issues might hang upon our journey. Miss Morstan's demeanour was as resolute and collected as ever. I endeavoured to cheer and amuse her by reminiscences of my adventures in Afghanistan; but, to tell the truth, I was myself so excited at our situation and so curious as to our destination that my stories were slightly involved. To this day she declares that I told her one moving anecdote as to how a musket looked into my tent at the dead of night, and how I fired a double-barrelled tiger cub at it. At first I had some idea as to the direction in which we were driving; but soon, what with our pace, the fog, and my own limited knowledge of London, I lost my bearings, and knew nothing, save that we seemed to be going a very long way. Sherlock Holmes was never at fault, however, and he muttered the names as the cab rattled through squares and in and out by tortuous by-streets.

»Rochester Row,« said he. »Now Vincent Square. Now we come out on the Vauxhall Bridge Road. We are making for the Surrey side, apparently. Yes, I thought so. Now we are on the bridge. You can catch glimpses of the river.«

We did indeed get a fleeting view of a stretch of the Thames with the lamps shining upon the broad, silent water; but our cab dashed on, and was soon involved in a labyrinth of streets upon the other side.

»Wordsworth Road,« said my companion. »Priory Road. Lark Hall Lane. Stockwell Place. Robert Street. Cold Harbor Lane. Our quest does not appear to take us to very fashionable regions.«

We had, indeed, reached a questionable and forbidding neighbourhood. Long lines of dull brick houses were only relieved by the coarse glare and tawdry brilliancy of public houses at the corner. Then came rows of two-storied villas each with a fronting of miniature garden, and then again interminable lines of new staring brick buildings,—the monster tentacles which the giant city was throwing out into the country. At last the cab drew up at the third house in a new terrace. None of the other houses were inhabited, and that at which we stopped was as dark as its neighbours, save for a single glimmer in the kitchen window. On our knocking, however, the door was instantly thrown open by a Hindoo servant clad in a yellow turban, white loose-fitting clothes, and a yellow sash. There was something strangely incongruous in this Oriental figure framed in the commonplace doorway of a third-rate suburban dwelling-house.

»The Sahib awaits you,« said he, and even as he spoke there came a high piping voice from some inner room. »Show them in to me, khitmutgar,« it cried. »Show them straight in to me.«